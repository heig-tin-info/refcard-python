\documentclass{article}

\usepackage[a4paper, landscape, margin=1cm]{geometry}
\usepackage{fontspec}
\usepackage[dvipsnames]{xcolor}
\usepackage[french]{babel}
\usepackage[fontsize=6.5pt]{scrextend}
\usepackage[T1]{fontenc}
\usepackage{multicol}
\usepackage{tabularx}
\usepackage{sectsty}
\usepackage{lmodern}
\usepackage{stix}
\usepackage{listings}
\usepackage{multirow}
\usepackage{titlesec}
\usepackage{fancyvrb,cprotect}
\usepackage{graphicx}

\input{revision}

\ifdefined\tinted
  \pagecolor{SpringGreen!60}
\fi

\setlength{\parskip}{0.2em}
\setlength{\parindent}{0em}

% Highlight configuration for C programming language
\lstset{
  language=python,
  breaklines=true,
  keywordstyle=\bfseries\color{black},
  basicstyle=\ttfamily\color{black},
  emphstyle={\em \color{gray}},
  commentstyle=\color{gray}\ttfamily,
  mathescape=true,
  keepspaces=true,
  showspaces=false,
  showtabs=true,
  tabsize=3,
  columns=fullflexible,
  aboveskip=0pt,
  belowskip=0pt
}

% Configuration
\renewcommand{\familydefault}{\sfdefault}

\allsectionsfont{\sffamily}

% No pages numbering
\pagenumbering{gobble}

\newlength\mybaselinestretch
\mybaselinestretch=0pt plus 0.02pt\relax
\addtolength{\baselineskip}{\mybaselinestretch}

\setlength\parindent{0pt}
\setlength\tabcolsep{1.5pt}
\setlength{\columnseprule}{0.4pt}

% Macros
\newcommand{\tab}{\hspace{2em}}
\newcommand{\etc}{\small \ldots}
\newcommand{\any}{$\hzigzag$~}
\newcommand{\spc}{$\mathvisiblespace$}
\newcommand{\cd}{\lstinline}

% Titles and paragraphs more compact
\titlespacing*{\section}{0em}{0.5em}{0.2em}
%\titlespacing*{\subsection}{0pt}{1em}{0pt}


\begin{document}

\begin{multicols*}{4}
\begin{tabularx}{\columnwidth}{lX}
    \raisebox{-\totalheight}{\includegraphics[width=1cm]{assets/heig-vd-black.pdf}} &
\begin{center}
  {\Large \bf Carte de référence Python 3.x} \\
  version \revision \ -- \revisiondate \\
\end{center}
\end{tabularx}
{
\scriptsize
Cette carte de référence peut être utilisée durant les travaux écrits
des cours utilisant \emph{python} à moins que le contraire ne soit explicitement formulé par l'enseignant. Cette carte comprend une liste non exhaustive des possibilités du langage Python 3 résumé en une feuille A4 recto-verso. Ce document est en partie inspiré de \emph{Learn X in Y}.

}
\section*{Différences entre Python 2.x et Python 3.x}
Python 2 est obsolète depuis le 1er janvier 2020.
\begin{lstlisting}
# Python 3.x                 # Python 2.x
print('hello')               print 'hello'
2/3 == 0.66                  2/3 = 0
"unicode"                    u"unicode"
range(1,2)                   xrange(1,2)
raise Exception("error")     raise "error"

from __future__ import print_function
\end{lstlisting}

\section*{Opérateurs et généralités}

\begin{lstlisting}
5.0 // 3                # => 1.0 Division entière
7 % 3                   # => 1 Modulo
2 ** 4                  # => 16 Exponentation
not True == False       # => True (Valeurs booléeenes)
True and False or False # => False
0 and 2                 # => 0 (Op. logiques sur int)
0 == False              # => True
1 != 1                  # => False (Inégalités)
1 < 2 < 3               # => True (Chaînage)
None                    # Un objet de type None
"etc" is None           # => False
None is None            # => True
bool(0)                 # => False
bool("")                # => False
\end{lstlisting}
\section*{Conteneurs}

\begin{tabularx}{\columnwidth}{llX}
Liste & \texttt{[1, 2, 3]} & Modifiable, ordonnée, indexée \\
Tuple & \texttt{(1, 2, 3)} & Non modifiable, ordonnée, indexée \\
Dict & \texttt{\{1: "a", 2: "b"\}} & Modifiable, non ordonné, indexé \\
Set & \texttt{\{1, 2, 3\}} & Modifiable, non ordonné, non indexé \\

\end{tabularx}

\section*{Initialisations}

\begin{lstlisting}
i = 42 + 2j       # Init une variable complexe
type(i)           # => <class 'complex'>
a = [1, 2, 3, 4]  # Nouvelle instance de liste
b = a             # Référence vers la même liste
b is a            # => True (b pointe sur a)
b == a            # => True (objets a et b égaux)
c = a[:]          # Copie de la liste
c is a            # => False (c pas référence sur a)
d = a.copy()      # Autre manière de copier une liste
\end{lstlisting}

\section*{Chaînes de caractères}

\begin{lstlisting}
"Hello, World!"        # Guillemets doubles
'Hello, World, too!'   # Guillemets simples
"""Multi
Ligne"""               # Utilisé pour le docstring
"Carottes " + "Cuites" # => "Carottes Cuites"
"A" "B"                # => "AB"
"Trois" * 3            # => "TroisTroisTrois"
"Webb"[0]              # => "W"
"{}!={}".format("Python", "Java") # => "Python!=Java"
"{a} {b}".format(a="foo", b="bar") # => "foo bar"
"{1} {0} {0} {1}".format("b", "a") # => "a b b a"
name = r"\r\ea"          # Caractères non échappés
"%s %d" % ("Hello", 42)  # => "Hello 42"
f"Bonjour {name}!"       # => "Bonjour Bob!"
print("Hello", end="!")  # => "Hello!"
print("Hello", )         # => "Hello" sans \n
print("Hello")           # => "Hello\n"
print("Error", file=sys.stderr) # sur stderr
','.join(["a", "b", "c"]) # => "a,b,c"
"foobar".upper() # => "FOOBAR"
"FOOBAR".lower() # => "foobar"
"foobar".capitalize() # => "Foobar"
"un chat gris".title() # => "Un chat gris"

\end{lstlisting}

\section*{Listes}
Tableau dynamique d'objets modifiable, ordonné et indexé.
\begin{lstlisting}
a = []              # Nouvelle instance de liste vide
len(a)              # => 0
a.append(1)         # a = [1]
a.append(2)         # a = [1, 2]
a.append(3, 4)      # a = [1, 2, 3, 4]
a.pop()             # a = [1, 2, 3], retourne 4
a[0]                # => 1
a[-1]               # => 3 (en partant de la fin)
a[42]               # => IndexError
a[0:3]              # => [1, 2, 3]
a[0:3:1]            # => [1, 2, 3]
a[0:]               # de 0 à la fin
a[:2]               # du début à 2
a[::2]              # Un élément sur 2
a[::-1]             # Inverse la liste
del a[0]            # a = [2, 3] (supprime 1)
b = [4] + [5, 6]    # b = [4, 5, 6]
a + b               # => [1, 2, 3, 4, 5, 6]
b.extend([8, 9])    # b = [4, 5, 6, 8, 9]
1 in a              # => False
a, b, c = [1, 2, 3] # a = 1, b = 2, c = 3
a, b = b, a         # a = 2, b = 1
a = [1, 3.15, "Hu", [1, 2]] # Liste composite
\end{lstlisting}

\section*{Tuples}
Liste non modifiable \emph{hashable} utilisable comme clé.
\begin{lstlisting}
a = (1, 2, 3)  # Nouvelle instance de tuple
a[0]           # => 1
a[0] = 42      # TypeError
hash(a)        # 529344067295497451
return 1, 2    # => (1, 2)
\end{lstlisting}

\section*{Dictionnaires}
\emph{hashmap} ou une collection de paires clé/valeurs.
\begin{lstlisting}
a = {}                      # New instance de dict
a["key"] = "value"          # Ajoute un élément
b = {"key": "value"}        # ... autre méthode
invalid = {[1,2]: "value"}  # TypeError
valid = {(1,2): "value"}    # Ok
list(a.keys())              # => ["key"]
list(a.values())            # => ["value"]
a['oops']                   # KeyError
a.get("oops")               # => None
a.get("oops", "default")    # => "default"
u = {'bob': 42, 'alice': 43}
u.items() # => dict_item([('bob', 42), ('alice', 43)])
\end{lstlisting}

\section*{Sets}
Collection de valeurs uniques, sans ordre, non indexé.
\begin{lstlisting}
a = set()      # Nouvelle instance de set
a.add(1)       # a = {1}
a.add(2)       # a = {1, 2}
a.add(2)       # a = {1, 2} Déjà dans a
b = {2, 5, 7}  # Nouvelle instance de set
a.union(b)     # => {1, 2, 5, 7}
a & b          # => {2} Intersection
a | b          # => {1, 2, 5, 7} Union
a - b          # => {1} Différence
1 in a         # => False
\end{lstlisting}

\section*{Structures de contrôle}
\begin{lstlisting}
x = 42
if x > 10: # Conditionnel if-elif-else
    print("x est supérieur à 10")
elif x > 5:
    print("x est supérieur à 5")
else:
    print("x est inférieur ou égal à 5")

for animal in ["chat", "chien", "oiseau"]:
    print(f"Un {animal} est un mammifère")

for i in range(4): print(i, end=" ") # 0 1 2 3
for i in range(4, 8): print(i, end=" ") # 4 5 6 7
for i in range(0, 10, 2): print(i, end=" ") # 0 2 4 6 8

animaux = ["cat", "dog", "fox"]
for i, animal in enumerate(animaux):
    print(f"{i}: {animal}") # 0: cat 1: dog 2: fox

d = {"c": 1, "b": 2, "a": 3}
for key in d: print(key) # c b a
for key in sorted(d): print(key) # a b c
for k, v in d.items(): print(k, v) # a 3 b 2 c 1
i = 0
while i < 4: print(i := i + 1) # 1 2 3 4

# Switch (Python >= 3.10)
status = 400
match status:
  case 200: print("OK")
  case 400: print("Bad Request")
  case 404: print("Not Found")
  case k
  case _: print("Status inconnu") # Default
\end{lstlisting}

\section*{Fonctions}
\begin{lstlisting}
def foo:
    pass # Pass est un mot clé qui ne fait rien

def add(a, b, c): # a, b, c sont des paramètres
    return a + b + c # Retourne la somme

def add(a, b, c=0): # Paramètre optionnel
    return a, b, c # Retourne un tuple

def add(a, b, *args): # Paramètre de type tuple
    return a + b + sum(args)

def display(*args):
    for arg in args: print(arg)

def display(**kwargs): # dict en entrée
    for key, value in kwargs.items(): print(key, value)

def bar(a, b=0, *args, **kwargs):
    print(a, b, args, kwargs)
bar(1, 2, 3, c=9, d=(1,2))
# => 1 2 (3,) {'c': 9, 'd': (1, 2)}

f = lambda x, y: x + y # Fonctions anonymes
f(1, 2)                # => 3
def iterate(x, f):
    for i in range(x):
        f(i)
iterate(3, lambda i: print(i)) # => 0 1 2

(lambda x: x > 2)(3)                 # => True
(lambda x, y: x ** 2 + y ** 2)(2, 3) # => 5

x = 0 # Scope global (mauvaise pratique)
def foo():
    global x                  # ... pas bien.
    x = 42
foo() # x => 42

def foo():                    # Fonctions imbriquées
    def bar():
        print("bar")
    bar()

def double_numbers(iterable): # Générateur
for i in iterable:
    yield i + i

# Annotation de fonction (Python >= 3.8)
def add(x: float, y: int) -> tuple:
    return (x, y, x + y)
\end{lstlisting}

\section*{Fichiers}
\begin{lstlisting}
# Ouverture d'un fichier
with open("file.txt", "r") as fp:
    for line in fp:
        print(line) # Affiche chaque ligne
# r: lecture seule, w: écriture seule, a: écriture à la suite
# b, mode binaire (rb, wb, ...)
# t, mode texte (rt, wt, ...)
# +, mode plus (r+, w+, ...)
# U, mode UTF-8 (rU, wU, ...)
fp = open("foo.txt", "w") # Mauvaise pratique
\end{lstlisting}

\section*{Exceptions}
\begin{lstlisting}
try:
    # Code susceptible d'échouer
except Exception as e:
    # Code exécuté en cas d'erreur
else:
    # Code exécuté en cas de succès
finally:
    # Code exécuté quelque soit le résultat
\end{lstlisting}

\section*{Compréhensions}
\begin{lstlisting}[mathescape]
[2 ** x for x in range(5)]    # => [1, 2, 4, 8, 16]
{x for x in range(5)}         # => {0, 1, 2, 3, 4}
{x: 5 ** x for x in range(2)} # => {0: 1, 1: 5}
[x for x in range(5) if x % 2 == 0] # => [0, 2, 4]

# Générateur sous forme de compréhension
values = (-x for x in [1,2,3,4,5])

[(temp * 9/5) + 32 for temp in ctemps] # ${^\circ}$C -> ${^\circ}$F
\end{lstlisting}

\section*{Math}
\begin{lstlisting}
import math
math.sqrt(16)         # => 4.0
math.sin(math.pi / 2) # => 1.0
math.perm(5, 3)       # => 120
math.factorial(25)    # => 15511210043330985984000000
math.gcd(3912, 4830)  # => 6
\end{lstlisting}

\section*{Classes}
\begin{lstlisting}
class Human: # Instance de type
    # Shared class attribute by all instances
    species = "Homo Sapiens"

    # Initialiseur (appelé à l'instanciation)
    def __init__(self, name):
        self.name = name # Propriété publique
        self._age = 0 # Propriété cachée
    def say(self, msg):
        print(f"{self.name}: {msg}")
    def sing(self):
        return 'yo... yo... microphone check...'
    @classmethod # Partagée par toutes les instances
    def get_species(cls):
        return cls.species
    @staticmethod # Méthode sans références
    def grunt():
        print("*grunt*")
    @property     # Getter
    def age(self):
        return self._age
    @age.setter   # Setter
    def age(self, value):
        if value < 0:
            raise ValueError("Age non valide")
        self._age = value
    @age.deleter # Suppresseur
    def age(self):
        raise AttributeError("Age non supprimable")
    # Methodes magiques
    def __str__(self):
        return f"{self.name} ({self.age})"
    def __repr__(self):
        return f"Human('{self.name}', {self.age})"
    def __len__(self):
        return len(self.name)

class Supehero(Human): # Héritage
    species = "Super Homme"

    def __init__(self, name, movie=False,
        superpowers=["strength", "bulletproofing"]):
        self.fictional = True
        self.movie = movie
        self.superpowers = superpowers
        super().__init__(name) # Human __init__

    def sing(self): # Surcharge de sing
        return 'Dun, dun, DUN!'

class Batman(Human, Superhero): # Héritage multiple
    def __init__(self, *args, **kwargs):
        Superhero.__init__(self, None, movie=True,
            superpowers=['Wealthy'], *args, **kwargs)
        Bat.__init__(self, *args, can_fly=False,
            **kwargs)

        # override the value for the name attribute
        self.name = 'Sad Affleck'
\end{lstlisting}

\section*{Décorateurs}
\begin{lstlisting}
from functools import wraps
def beg(target_function):
    @wraps(target_function)
    def wrapper(*args, **kwargs):
        msg, say_please = target_function(*args, **kwargs)
        if say_please:
            return "{} {}".format(msg, "Please! I am poor :(")
        return msg
    return wrapper

@beg
def say(say_please=False):
    msg = "Can you buy me a beer?"
    return msg, say_please

print(say())                 # Can you buy me a beer?
print(say(say_please=True))  # Please! I am poor :(
\end{lstlisting}

\section*{Loggeur et traceurs}
\begin{lstlisting}
import logging
logging.warning('Watch out!')  # Sur la console
logging.info('I told you so')  # N'affiche rien

logging.basicConfig(filename='example.log',
    encoding='utf-8', level=logging.DEBUG)
logging.debug('Un message')

logger = logging.getLogger(__name__)
logger.setLevel(logging.DEBUG)

formatter = logging.Formatter('%(asctime)s - %(name)s'
    ' - %(levelname)s - %(message)s')
logger.setFormatter(formatter)
logger.addHandler(logging.StreamHandler())

\end{lstlisting}

\section*{Singleton}
\begin{lstlisting}
class Singleton(type):
_instances = {}
def __call__(cls, *args, **kwargs):
    if cls not in cls._instances:
        cls._instances[cls] = super(Singleton, cls).__call__(*args, **kwargs)
    return cls._instances[cls]
class MyClass(BaseClass, metaclass=Singleton):
    pass
\end{lstlisting}

\section*{Opérations sur les listes}
\begin{lstlisting}
enumerate(liste) # Renvoie un tuple (index, element)
list(enumerate(liste)) # Renvoie une liste de tuples
zip([1,2,3], [4,5,6]) # => [(1, 4), (2, 5), (3, 6)]
u = enumerate(zip([1,2,3], [4,5,6]))
# => [(0, (1, 4)), (1, (2, 5)), (2, (3, 6))]
for i, (x, y) in u:
    print(i, x, y)

filter(lambda x: x % 2 == 0, [1,2,3,4,5]) # => [2, 4]
\end{lstlisting}

\section*{Import et modules}
\begin{lstlisting}
from math import sqrt, prod  # Bonne pratique
from math import *           # Mauvaise pratique

# Comportement par défaut à l'appel du script
if __name__ == '__main__':
    main()
\end{lstlisting}

\subsection*{Numpy (calcul scientifique)}
\begin{lstlisting}
>>> import numpy as np
>>> a = np.arange(15).reshape(3, 5)
>>> a
array([[ 0,  1,  2,  3,  4],
       [ 5,  6,  7,  8,  9],
       [10, 11, 12, 13, 14]])
>>> a.shape                 # (3, 5)
>>> a.ndim                  # 2
>>> a.dtype.name            # 'int64'
>>> a.itemsize              # 8
>>> a.size                  # 15
>>> type(a)                 # <class 'numpy.ndarray'>
>>> np.array([6, 7, 8]) * 2 # array([12, 14, 16])
>>> np.zeros((3, 5))        # None
>>> np.zeros((3, 5), dtype=np.int16)
array([[0, 0, 0, 0, 0],
       [0, 0, 0, 0, 0],
       [0, 0, 0, 0, 0]])
>>> np.array([[1, 2], [3, 4]], dtype=complex)
array([[1.+0.j, 2.+0.j],
       [3.+0.j, 4.+0.j]])
>>> np.linspace(0, 2, 6)
array([0. , 0.4, 0.8, 1.2, 1.6, 2. ])
\end{lstlisting}

\subsection*{Matplotlib (graphiques)}
\begin{lstlisting}
>>> import matplotlib.pyplot as plt
>>> x = np.linspace(0, 2, 100)
>>> y = np.sin(2 * np.pi * x)
>>> plt.plot(x, y)
[<matplotlib.lines.Line2D object at 0x7f8b8b8b9b90>]
>>> plt.ylabel('some numbers')
>>> plt.show()
\end{lstlisting}

\subsection*{Functools (fonctions spéciales)}
\begin{lstlisting}
import functools
@cache
def factorial(n):
    return n * factorial(n-1) if n else 1

reduce(lambda x, y: x * y, [1, 2, 3, 4, 5]) # => 120
import operator
functools.reduce(operator.mul, [1, 2, 3])   # => 6
\end{lstlisting}

\subsection*{Itertools (itérateurs)}
\begin{lstlisting}
chain(*[range(3), range(3)]) # => 0, 1, 2, 0, 1, 2
combinations('ABCD', 2) # => AB AC AD BC BD CD
product('ABCD', 'xy') # => Ax Ay Bx By Cx Cy Dx Dy
permutations('ABCD', 2)
# => AB AC AD BA BC BD CA CB CD DA DB DC
combinations_with_replacement('ABCD', 2)
# => AA AB AC AD BB BC BD CC CD DD
\end{lstlisting}

\subsection*{Collections (conteneurs d'abstraction)}
\begin{lstlisting}
from collections.abc import Mapping
class DefaultDict(Mapping):
    def __init__(self, default_factory=None, *args, **kwargs):
        self.default_factory = default_factory
        self.store = dict()
        self.update(dict(*args, **kwargs))
    def __getitem__(self, key):
        if key not in self.store:
            self.store[key] = self.default_factory()
        return self.store[key]
    def __iter__(self):
        return iter(self.store)
    def __len__(self):
        return len(self.store)
    def __repr__(self):
        return 'DefaultDict(%s, %s)' % (self.default_factory, self.store)
\end{lstlisting}

\subsection*{Click (interfaces en ligne de commande)}
\begin{lstlisting}
@click.command
@click.option('--name', prompt='Your name', help='')
@click.option('--age', prompt='Your age', help='')
def hello(name, age):
    click.echo('Hello %s!' % name)
    click.echo('You are %s years old.' % age)
hello()
\end{lstlisting}

\subsection*{Expressions régulières}
\begin{lstlisting}
import re
re.findall(r'\bf[a-z]*', 'which foot or hand fell fastest')
['foot', 'fell', 'fastest']
re.split(r'\s*', 'a b c  d') # ['a', 'b', 'c', 'd']
re.sub(r'[aeiouy]', '', 'champion') # 'chmprn'
\end{lstlisting}

\section*{Modules}
\begin{description}
\item[beautifulsoup4] pour manipuler des pages web
\item[flask] framework pour créer des applications web
\item[functools] pour manipuler des fonctions
\item[itertools] pour manipuler des itérateurs
\item[jinja] moteur de template
\item[matplotlib] pour créer des graphiques
\item[numpy] pour manipuler des tableaux
\item[opencv] pour manipuler des images
\item[pandas] pour manipuler des données
\item[pillow] pour manipuler des images
\item[pint] pour manipuler des unités
\item[pytest] pour tester des applications
\item[requests] pour manipuler des requêtes HTTP
\item[scipy] pour manipuler des fonctions mathématiques
\item[seaborn] pour manipuler des graphiques
\item[pickle, yaml, json] pour manipuler des fichiers sérialisés
\item[click] pour des interfaces en ligne de commande
\end{description}

\subsection*{PEP (Python Enhancement Proposals)}
\begin{description}
    \item[PEP 8] Formattage du code Python
    \item[PEP 20] Zen of Python
    \item[PEP 257] Docstring Conventions (commentaires)
    \item[PEP 622] Structural Pattern Matching (match/case)
    \item[PEP 695] Syntaxe du type hinting
    \item[PEP 701] Syntaxe des fstrings
\end{description}

\end{multicols*}

\end{document}
